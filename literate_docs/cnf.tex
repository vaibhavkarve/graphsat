% Created 2021-03-18 Thu 09:44
% Intended LaTeX compiler: pdflatex
\documentclass[11pt]{article}
\usepackage[utf8x]{inputenc}
\usepackage[T1]{fontenc}
\usepackage{graphicx}
\usepackage{grffile}
\usepackage{longtable}
\usepackage{wrapfig}
\usepackage{rotating}
\usepackage[normalem]{ulem}
\usepackage{amsmath}
\usepackage{textcomp}
\usepackage{amssymb}
\usepackage{capt-of}
\usepackage{hyperref}
\usepackage{minted}
\usepackage{hyperref}
\hypersetup{colorlinks, linkcolor={blue!100!black}, citecolor={blue!50!black}, urlcolor={blue!80!black}}

\usepackage[nottoc,notlot,notlof]{tocbibind}

\usepackage{mathtools}
\usepackage{amssymb}
\usepackage[utf8x]{inputenc}
\usepackage{minted}

% Change to a font that has all the glyphs. This option necessitates
% XeLaTeX or LuaLaTeX as the compiler engine.
\usepackage{fontspec}
\setmonofont{DejaVu Sans Mono}[Scale=MatchLowercase]


\setminted{fontsize=\small}
\setminted{stripnl=false}
\usepackage{amsthm}
\usepackage[margin=1in]{geometry}
\usepackage{parskip}
\usepackage{todonotes}
%\usepackage{fontspec}  % use any system font with LuaLaTeX or XeLaTeX only
%\usepackage{unicode-math}
%\usepackage[mathletters]{ucs}

\usemintedstyle{emacs}

\newtheorem{thm}{Theorem}
\newtheorem{lemma}{Lemma}
\newtheorem{cor}{Corollary}


\setminted{python3=true}
\setminted{labelposition=topline}


%%% Local Variables:
%%% mode: latex
%%% End:

\author{Vaibhav Karve}
\date{\today}
\title{cnf.py\\\medskip
\large Constructors and functions for sentences in conjunctive normal form.}
\hypersetup{
 pdfauthor={Vaibhav Karve},
 pdftitle={cnf.py},
 pdfkeywords={},
 pdfsubject={},
 pdfcreator={Emacs 26.3 (Org mode 9.1.9)},
 pdflang={English}}
\begin{document}

\maketitle
\setcounter{tocdepth}{2}
\tableofcontents





\section{Introduction}
\label{sec:orga8ba48d}
This module defines Cnf (boolean formulas in conjunctive normal form) as a
python class and creates an API of functions for defining Cnfs and computing
with them. We define variables, literals, clauses, Cnfs and assignements in \S
\ref{sec:org27c3b34} and functions that manipulate them in \S\S
\ref{sec:org95949d5}--\ref{sec:org3162dd0}.

This module has been written in the functional-programming style whose
principles can be summarized as ---
\begin{enumerate}
\item Avoid classes as much as possible. Prefer defining functions instead.
\item Write small functions and then compose/map/filter them to create more complex
functions (using the \href{https://docs.python.org/3/library/functools.html}{\texttt{functools}} library).
\item Use lazy evaluation strategy whenever possible (using the \href{https://docs.python.org/3/library/itertools.html}{\texttt{itertools}} library).
\item Add type hints wherever possible (checked using the \href{https://mypy.readthedocs.io/en/stable/}{\texttt{mypy}} static type-checker).
\item Add unit-tests for each function (checked using the \href{https://docs.pytest.org/en/latest/}{\texttt{pytest}} framework).
\end{enumerate}

The end-goal of this module is to equip us with a Cnf-calculator using python
syntax. We illustrate this using the following example. Consider the following
statement --- assigning the variable \(x_2\) to true and \(x_3\) to false in
\((x_1\vee\neg x_2\vee x_4)\wedge(x_1 \vee \neg x_3\vee x_5)\) gives us
\(x_1\vee x_4\).

Using notation defined in the previous[??] chapter, we can verify this
statement by carrying out the following calculation.
\begin{flalign*}
 1\overline{2}4,1\overline{3}5\left[\,2, \overline{3}\,\right]\;
   &=\; 1\bot4,1\overline{3}5\left[\,\overline{3}\,\right] &\\
   &=\; 14,1\overline{3}5\left[\,\overline{3}\,\right] &\\
   &=\; 14,1\top5 &\\
   &=\; 14,\top &\\
   &=\; 14.
\end{flalign*}

The same calcultion can also be carried out in python using the
functions defined in this module. The output is included below the
codeblock.
\begin{minted}[frame=lines,label= (python3.9) (scratch) <<example>>]{python}
from cnf import assign, Assignment, cnf, Cnf, TRUE, FALSE

x : Cnf = cnf([[1, -2, 4], [1, -3, 5]])
a : Assignment = {2: TRUE, 3: FALSE}

return assign(x, a)
\end{minted}

\footnotesize
\uline{Output:}
\begin{verbatim}
Cnf({Clause({1, 4})})
\end{verbatim}
\normalsize

All codeblocks in this chapter have been formatted in a manner similar the one
shown above. The top-margin of the codeblock starts by indicating the
programming language used in the block (\texttt{python3.9} in the example). Shell
scripts for example will have \texttt{sh} in that position instead.

The second of three slots represents the file of which this codeblock is a
part. For example, a codeblock that has \texttt{cnf.py} in this slot will be a part of
the \texttt{cnf.py} file. This helps with persistence of variables and definitions
defined the file -- codeblocks in \texttt{cnf.py} can refer to other variables from
different codeblocks that are also part of \texttt{cnf.py}. We use the name \texttt{scratch}
to represent a throw-away file. In the example above, the \texttt{scratch} label means
that this was a one-time calculation that we do not wish to store in a file.

The third label in the top-margin of the codeblock represents a unique
identifier of the codeblock (\texttt{<<example>>} in the example above). This
identifier is used to keep track of the codeblock we are referring to as well as
to refer to it inside other codeblocks (via the process called \emph{tangling} in the
literate programming terminology).


\subsection{Definitions}
\label{sec:org0d0b4e5}
We start by summarizing the relevant definitions in top-down
order. Details for each definition and accompanying implementations
can be found in the relevant subsections.
\begin{itemize}
\item A \hyperref[sec:orgbb8fffd]{Cnf} is a conjunction of one or more Clauses. It is implemented as a subclass
of frozensets of Clauses in \S \ref{sec:orgbb8fffd}.
\item A \hyperref[sec:org3708534]{Clause} is a disjunction of one or more Lits. It is implemented as a subclass
of frozensets of Lits in \S \ref{sec:org3708534}.
\item A \hyperref[sec:org06a8c37]{Lit} is either a Variable, the negation of a Variable, or a Bool.  These are
usually called \emph{Literals} in CS literature but this sounds too close to
\texttt{typing.Literal} which is already defined in python and refers to a different,
unrelated object. It is implemented as a subtype of python's built-in data type
\texttt{int} in \S \ref{sec:org06a8c37}.
\item A \hyperref[sec:orgdf49146]{Variable} is any element from a fixed countably infinite set of symbols that
can be assigned the value of Bool in a boolean expression. We implement it as
a subtype of \texttt{int}, with the additional restriction that only positive \texttt{int}
values are allowed. The implementation is explained in \S \ref{sec:orgdf49146}.
\item A \hyperref[sec:orgc16e7e4]{Bool} is a set of two reserved the symbols called \texttt{TRUE} and \texttt{FALSE} (not to
be confused with \texttt{bool}, \texttt{True} and \texttt{False}).  Justification of the need for
defining such a type distinct from \texttt{bool} can be found in \S \ref{sec:orgc16e7e4}. That
section also details the implementation of Bool as a subclass of Lits.
\end{itemize}

\newpage
\subsection{Overview of \texttt{cnf.py}}
\label{sec:orgd32c73b}
\begin{minted}[frame=lines,label= (python3.9) (cnf.py) <<cnf.py-full>>]{python}
#!/usr/bin/env python3.9
"""Constructors and functions for sentences in conjunctive normal form (Cnf)."""

# Imports
# =======
<<imports>>

# Classes and Types
# =================
<<Variable>>
<<Lit>>
<<Bool>>
<<TRUE_and_FALSE>>  # Instances of Bool, needed to define Clause and Cnf
<<Clause>>
<<Cnf>>
<<Assignment>>

# Constructor Functions
# =====================
<<variable>>
<<lit>>
<<clause>>
<<cnf>>

# Helpful Constants
# =================
<<constants>>  # not documented, for internal use only

# Basic Functions
# ===============
<<neg>>
<<absolute_value>>
<<lits>>

# Functions for Simplification
# ============================
<<tautologically_reduce_clause>>
<<tautologically_reduce_cnf>>

# Functions for Assignment
# ========================
<<assign_variable_in_lit>>
<<assign_variable_in_clause>>
<<assign_variable_in_cnf>>
<<assign>>

if __name__ == '__main__':
    <<standalone_run_commands>>
\end{minted}

\subsection{Imports and dependencies}
\label{sec:orgb7b05e5}
We start by importing the required packages.  We will be using the \href{https://docs.python.org/3/library/typing.html}{\texttt{typing}} and
\href{https://docs.python.org/3/library/collections.abc.html}{\texttt{collections.abc}} libraries to add type-hints throughout our code.  This acts as
a kind of documentation while also catching type-errors when we use it in
conjunction with an external static type-checker like \href{https://mypy.readthedocs.io/en/stable/}{\texttt{mypy}}.

\begin{minted}[frame=lines,label= (python3.9)  (cnf.py) <<imports>>]{python}
from collections.abc import Set, Callable, Collection, Iterator, Mapping
import functools as ft
from typing import final, Final, Mapping, NewType, Union

from loguru import logger
\end{minted}

\section{Types and their constructor functions}
\label{sec:org27c3b34}
We now implement each of the definitions given in \S \ref{sec:org0d0b4e5} as types.  For
each type we will also define a unique constructor function.

\subsection{Variables}
\label{sec:orgdf49146}
A \emph{Variable} is any element from a fixed countably infinite set of symbols that
can be assigned the value of \hyperref[sec:orgc16e7e4]{Bool} in a boolean expression.  We use python's
built-in \texttt{int} type as our countably infinite set.  Using \href{https://docs.python.org/3/library/typing.html\#typing.NewType}{\texttt{typing.NewType}}, we
can declare Variable to be a subtype of \texttt{int}.  This means that any \texttt{int} \(n\)
can be cast into a term of type Variable as \texttt{Variable(n)}. In type-theoretic
notation, we can write the relation between Variable and \texttt{int} as
\(\textrm{Variable} \sqsubset \texttt{int}\).

\begin{minted}[frame=lines,label= (python3.9)  (cnf.py) <<Variable>>]{python}
Variable = NewType('Variable', int)
Variable.__doc__ = """`Variable` is a subtype of `int`."""
\end{minted}

In fact, we only want the positive integers to denote Variables.  However, this
restriction cannot be encoded when defining the type itself and so we also write
a constructor function \texttt{variable} to check for positivity.
\begin{minted}[frame=lines,label= (python3.9)  (cnf.py) <<variable>>]{python}
def variable(positive_int: int) -> Variable:
    <<variable-docstring>>
    if positive_int <= 0:
	raise ValueError('Variable must be a positive integer.')
    return Variable(positive_int)
\end{minted}


We note that this function is injective and idempotent.
\begin{minted}[frame=lines,label= (python3.9)  (test/test\_cnf.py) <<test\_variable>>]{python}
def test_variable():
    assert variable(1) == 1
    assert variable(11) == 11
    assert variable(variable(2)) == variable(2)  # Test for idempotence
    pytest.raises(ValueError, variable, 0)
\end{minted}


\subsection{Lits}
\label{sec:org06a8c37}
A \emph{Lit} is either a Variable, its negation, or a \hyperref[sec:orgc16e7e4]{Bool}. We have not yet defined
\texttt{Bool}, but for now we can simply assume that it is a special type of \texttt{int}
(justification for this can be found in \S \ref{sec:orgc16e7e4}). Since both \texttt{Variable} and \texttt{Bool}
are subtypes of \texttt{int}, we can define \texttt{Lit} in a similar.  The only caveat to
this is that we want \texttt{Lit} to act as a subscriptable argument later, allowing us
to define \texttt{Clause} as a being a subtype/subclass of frozen sets of \texttt{Lit}. To
make \texttt{Lit} subscriptable, we define it as a \textbf{subclass} of \texttt{int} instead of a
\textbf{subtype}.

\begin{minted}[frame=lines,label= (python3.9)  (cnf.py) <<Lit>>]{python}
class Lit(int):
    """`Lit` is a subclass of `int`. It has no other special methods."""
\end{minted}

Next, we define a constructor function for \texttt{Lit}, called \texttt{lit}.  On \texttt{Bools}, we
want the constructor function to act as the identity function, while on an
integer argument, it should check that the integer is nonzero (thus ensuring
that it is either a \texttt{Variable} or its negation).  To implement such a piece-wise
definition we use \href{https://docs.python.org/3/library/functools.html\#functools.singledispatch}{\texttt{functools.singledispatch}}.  This allows us to pick the
correct variant of \texttt{lit} based on the type of its first (and only) argument.


\begin{minted}[frame=lines,label= (python3.9) (cnf.py) <<lit>>]{python}
@ft.singledispatch
def lit(int_or_bool: Union[int, Bool]) -> Lit:
    <<lit-docstring>>
    raise TypeError('Lit must be either Bool or int.')

@lit.register
<<lit_bool>>

@lit.register
<<lit_int>>
\end{minted}

where we have the following functions that are single-dispatched based
on the type of the first argument.

\begin{minted}[frame=lines,label= (python3.9) (cnf.py) <<lit\_bool>>]{python}
def lit_bool(arg: Bool) -> Lit:
    """Return as is because Bool is already a subtype of Lit."""
    return arg
\end{minted}

\begin{minted}[frame=lines,label= (python3.9) (cnf.py) <<lit\_int>>]{python}
def lit_int(arg: int) -> Lit:
    """Cast to Lit."""
    if arg != 0:
	return Lit(arg)
    raise ValueError('Lit must be a nonzero integer.')
\end{minted}

As mentioned in its docstring below, \texttt{lit} is an idempotent function. To
ensure this, we write the following set of tests.  (While we have not
yet defined \texttt{Bool.TRUE} and \texttt{Bool.FALSE}, their meaning in the
following assert statements should be obvious from context).

\todo{fix unicode}
\begin{minted}[frame=lines,label= (python3.9) (cnf.py) <<lit-docstring>>]{python}
"""Constructor-function for Lit type.

By definition, a `Lit` is in the set ℤ ∪ {`TRUE`, `FALSE`} ∖ {0}`.
This function is idempotent.
"""
\end{minted}

\begin{minted}[frame=lines,label= (python3.9) (test/test\_cnf.py) <<test\_lit>>]{python}
def test_lit():
    assert lit(1) == 1
    assert lit(-1) == -1
    assert lit(11) ==  11
    assert lit(TRUE) == TRUE
    assert lit(FALSE) == FALSE
    assert lit(lit(2)) == lit(2)          # Test for idempotence.
    assert lit(lit(TRUE)) == lit(TRUE)    # Test for idempotence.
    assert lit(lit(FALSE)) == lit(FALSE)  # Test for idempotence.

    pytest.raises(ValueError, lit, 0)
\end{minted}

\subsection{Bools}
\label{sec:orgc16e7e4}
\emph{Bool} refers to two reserved symbols -- \texttt{TRUE} and \texttt{FALSE}. Note that
python already has a \texttt{bool} type (with elements \texttt{True} and \texttt{False}).
However, \texttt{bool} is actually implemented as a subclass of the \texttt{int} type.
This has some unintended consequences when considering data structures that
contain both integers and elements due to the fact that \texttt{True} and \texttt{False}
are just aliases for the integers \texttt{1} and \texttt{0} respectively. This means that
python will always simplify the set \mintinline[frame=lines]{python}{{1, 2, 3, True}} to
\mintinline[frame=lines]{python}{{1, 2, 3}} since it treats \texttt{True} as a duplicate of \texttt{1}. This is
a big problem and the primary reason why we define Bools to be distinct
from \texttt{bool}.

We want Bool to be a type that is a subclass of \texttt{int} and not just a
subtype. In fact, since all Bools are Lits, we also want Bool to be a
subtype of Lit. Thus we define Bool as shown below with special methods
\texttt{\_\_str\_\_}, \texttt{\_\_hash\_\_}, and \texttt{\_\_eq\_\_}. Additionally, we define the \texttt{\_\_repr\_\_}
method to be identical to the \texttt{\_\_str\_\_} method. All other methods of Bool
are inherited from the \texttt{int} parent type.

\begin{minted}[frame=lines,label= (python3.9) (cnf.py) <<Bool>>]{python}
@final
class Bool(Lit):
    """`Bool` is a subclass of `Lit`.

    It overrides the ``__str__``, ``__repr__``, ``__hash__`` and ``__eq__``
    methods inherited from :obj:`int` (and from Lit).
    """
    <<Bool-str>>
    __repr__ = __str__
    <<Bool-hash>>
    <<Bool-eq>>
\end{minted}

We now give detailed definitions for each of Bool's special methods. The
\texttt{\_\_str\_\_} method allows us to change the value displayed when we print the a
Bool. Even though python will internally store Bools as \texttt{0} and \texttt{1}, when asked,
it will still print the values as \texttt{<Bool: TRUE>} and \texttt{<Bool: FALSE>}
respectively.  Any other numeric value raises a \texttt{ValueError}.

\begin{minted}[frame=lines,label= (python3.9) (cnf.py) <<Bool-str>>]{python}
def __str__(self) -> str:
    """Bool(0) and Bool(1) are treated as constant values labeled FALSE and TRUE."""
    if self.__int__() == 0:
	return '<Bool: FALSE>'
    if self.__int__() == 1:
	return '<Bool: TRUE>'
    raise ValueError('In-valid Bool value encountered.')
\end{minted}

To prevent python from simplifying \mintinline[frame=lines]{python}{{1, TRUE}} to \mintinline[frame=lines]{python}{{1}}, we modify
its hash value. We set the hash of \texttt{TRUE} to \mintinline[frame=lines]{python}{hash(str(TRUE))}, which
simplifies to \(4422589586725646474\). Similarly, the hash of \texttt{FALSE} is set to
\(6211323488567046769\).
\begin{minted}[frame=lines,label= (python3.9) (cnf.py) <<Bool-hash>>]{python}
def __hash__(self) -> int:
    """Ensure that ``hash(Bool(n))`` doesn't clash with ``hash(n)``."""
    return hash(str(self))
\end{minted}

Moreover, we redefine the \texttt{\_\_eq\_\_} method to now use hash values as a means to
establish equality.  This makes \texttt{Bool(n)} unequal to \texttt{n}.
\begin{minted}[frame=lines,label= (python3.9) (cnf.py) <<Bool-eq>>]{python}
def __eq__(self, other: object) -> bool:
    """Make ``Bool(n)`` unequal to ``n``."""
    return hash(self) == hash(other)
\end{minted}


Finally, having defined the class of Bool, we now define exactly two
instances of Bool. In type-theoretic language, we can say that Bool is an
enumerated type (since we enumerate all its instances explicitly).
Furthermore, to prevent the user from accidentally (or maliciously)
redefining the values of \texttt{TRUE} and \texttt{FALSE}, we use \href{https://docs.python.org/3/library/typing.html?\#typing.Final}{typing.Final} to disable
overriding and re-assigning of these constants. We also mark the class
definition of Bool itself with the \href{https://docs.python.org/3/library/typing.html?\#typing.final}{@typing.final} decorator in order to
prevent this class from being sub-classed by the user.
\begin{minted}[frame=lines,label= (python3.9) (cnf.py) <<TRUE\_and\_FALSE>>]{python}
#: ``TRUE = Bool(1)``, a final instance of Bool.
TRUE: Final = Bool(1)
#: ``FALSE = Bool(0)``, a final instance of Bool.
FALSE: Final = Bool(0)
\end{minted}

Lastly, to check that the Bool values indeed possess all the desired properties,
we add unit-tests.
\begin{minted}[frame=lines,label= (python3.9) (test/test\_cnf.py) <<test\_Bool>>]{python}
def test_Bool():
    assert TRUE == TRUE               # check for consistency
    assert TRUE not in [1, 2, 3]      # check that eq is working
    assert TRUE in {1, 2, 3, TRUE}    # check that eq and hash are working
    assert isinstance(TRUE, Bool)     # check that python recognizes the class correctly
    assert not isinstance(TRUE, bool) # check that Bool and bool are district
\end{minted}


\subsection{Clauses}
\label{sec:org3708534}
Next, we wish to define a \emph{Clause} to be a container of Lits. The choice of
container is driven by three properties we desire in Clauses --
\begin{enumerate}
\item We do not care for repetitions in the clause. This is in keeping with the
identity \(x\vee x=x\).
\item We also do not care about the ordering of elements in a clause. This is in
keeping with the fact that disjunction is commutative.
\item Looking ahead, we will want to put clauses inside other containers to form
Cnf. Thus it is desirable that clauses be hashable (and therefore immutable)
objects.
\end{enumerate}
These properties suggest that we should have our containers be frozensets
(python's unordered, immutable, hashable sets).  We define Clause as a subclass
instead of just a subtype, allowing us to also override its \texttt{\_\_str\_\_} method to
offer more human-readable printouts.

\begin{minted}[frame=lines,label= (python3.9) (cnf.py) <<Clause>>]{python}
class Clause(frozenset[Lit]):  # pylint: disable=too-few-public-methods
    """`Clause` is a subclass of `frozenset[Lit]`."""
    def __str__(self) -> str:
	"""Pretty print a Clause after sorting its contents."""
	sorted_clause: list[Lit] = sorted(self, key=absolute_value)
	return '(' + ','.join(map(str, sorted_clause)) + ')'
\end{minted}

Having defined a class, we next define a constructor function \texttt{clause} for said
class. The function ensures that a Clause is always a nonempty collection of
Lits. The function is also guaranteed to be idempotent.
\begin{minted}[frame=lines,label= (python3.9) (cnf.py) <<clause>>]{python}
def clause(lit_collection: Collection[int]) -> Clause:
    <<clause-docstring>>
    if not lit_collection:
	raise ValueError(f'Encountered empty input {list(lit_collection)}.')
    return Clause(frozenset(map(lit, lit_collection)))
\end{minted}

Finally, we add in the doc-string describing the function we have just defined
and we add some unit-tests to for the function as well.
\begin{minted}[frame=lines,label= (python3.9) (cnf.py) <<clause-docstring>>]{python}
"""Constructor-function for Clause type.

By definition, a `Clause` is a nonempty frozenset of Lits. This function is idempotent.

Args:
   lit_collection (:obj:`Collection[int]`): a nonempty collection of ints.

Return:
   Check that each element in the collection satisfies axioms for being a Lit and then cast to
      Clause.

Raises:
   ValueError: if ``lit_collection`` is an empty collection.

"""
\end{minted}


\begin{minted}[frame=lines,label= (python3.9) (test/test\_cnf.py) <<test\_clause>>]{python}
def test_clause():
    assert clause([1, 2, -3]) == {1, 2, -3}    # check for correct type
    assert clause([1, -1, 2]) == {1, -1, 2}    # +ve and -ve Lits are treated as distinct
    assert clause([TRUE]) == {TRUE}            # TRUE can be part of a Clause
    assert clause([FALSE]) == {FALSE}          # FALSE can be part of a Clause
    assert clause([1, TRUE]) == {1, TRUE}      # TRUE is distinct from 1 in a Clause
    assert clause([-1, FALSE]) == {-1, FALSE}  # FALSE is distinct from -1 in a Clause
    assert clause([1, TRUE, FALSE]) == {1, TRUE, FALSE}  # TRUE and FALSE can both appear

    # Tests for idempotence
    assert clause(clause([1, 2, -3])) == clause([1, 2, -3])
    assert clause(clause([TRUE])) == clause([TRUE])
    assert clause(clause([FALSE])) == clause([FALSE])

    pytest.raises(ValueError, clause, [])
\end{minted}

\subsection{Cnfs}
\label{sec:orgbb8fffd}
We define a \emph{Cnf} to be a container of Clauses.  The choice of container is
driven by three properties we desire in Cnfs --
\begin{enumerate}
\item We do not care for repetitions in the Cnf. This is in keeping with the
identity \(x\wedge x=x\).
\item We also do not care about the ordering of elements in a Cnf. This is in
keeping with the fact that conjunction is commutative.
\item Looking ahead, we will want to put Cnfs inside other containers when
performing calculations. Thus it is desirable that Cnfs be hashable (and
therefore immutable) objects.
\end{enumerate}
These properties are exactly the ones we desired for Clauses, so we
similarly select frozensets as our containers.  We define Cnf as a subclass
instead of just a subtype, allowing us to also override its \texttt{\_\_str\_\_}
method to offer more human-readable printouts.

\begin{minted}[frame=lines,label= (python3.9) (cnf.py) <<Cnf>>]{python}
class Cnf(frozenset[Clause]):  # pylint: disable=too-few-public-methods
    """`Cnf` is a subclass of `frozenset[Clause]`."""

    def __str__(self) -> str:
	"""Pretty print a Cnf after sorting its sorted clause tuples."""
	sorted_cnf: list[Clause]
	sorted_cnf = sorted(self, key=lambda clause_: sum([lit < 0 for lit in clause_]))
	sorted_cnf = sorted(sorted_cnf, key=len)

	cnf_tuple: Iterator[str] = map(str, map(clause, sorted_cnf))
	return ''.join(cnf_tuple)
\end{minted}

We define a constructor function \texttt{cnf} for this class. This function
ensures that a Cnf is always a nonempty collection of Clauses. This
constructor function is also guaranteed to be idempotent.

\begin{minted}[frame=lines,label= (python3.9) (cnf.py) <<cnf>>]{python}
def cnf(clause_collection: Collection[Collection[int]]) -> Cnf:
    <<cnf-docstring>>
    if not clause_collection:
	raise ValueError(f'Encountered empty input {list(clause_collection)}.')
    return Cnf(frozenset(map(clause, clause_collection)))
\end{minted}

Finally, we add in a docstring and some unit tests for the constructor function.
\begin{minted}[frame=lines,label= (python3.9) (cnf.py) <<cnf-docstring>>]{python}
"""Constructor-function for Cnf type.

By definition, a `Cnf` is a nonempty frozenset of Clauses. This function is idempotent.

Args:
   clause_collection (:obj:`Collection[Collection[int]]`): a nonempty collection (list, tuple,
      set, frozenset) of nonempty collections of integers or Bools.

Return:
   Check that each element in the collection satisfies axioms for being a Clause and then cast to
      Cnf.

Raises:
   ValueError: if ``clause_collection`` is an empty collection.
"""
\end{minted}


\begin{minted}[frame=lines,label= (python3.9) (test/test\_cnf.py) <<test\_cnf>>]{python}
def test_cnf():
    fs = frozenset  # a temporary alias for frozenset

    # Generic example use-case
    assert cnf([[1, 2, -3], [-4, 5]]) == {fs([1, 2, -3]), fs([-4, 5])}

    # Test for removing repetitions
    assert cnf([[1, 1, -1], [1, -1]]) == {fs([1, -1])}

    # Cnf with TRUE and FALSE inside a Clause
    assert cnf([[1, 2, TRUE], [3, FALSE]]) == {fs([1, 2, TRUE]), fs([3, FALSE])}

    # Cnf with a Bool-only Clause
    assert cnf([[1, 2, 3], [FALSE], [TRUE]]) == {fs([1, 2, 3]), fs([FALSE]), fs([TRUE])}
    assert cnf([[TRUE], [TRUE, TRUE]]) == {fs([TRUE])}

    # Single-Lit-single-Clause Cnf
    assert cnf([[1]]) == {fs([1])}
    assert cnf([[-1]]) == {fs([-1])}

    # Test for idempotence.
    assert cnf(cnf([[1, 2, 3], [-4, 5]])) == cnf([[1, 2, 3], [-4, 5]])

    pytest.raises(ValueError, cnf, [])
    pytest.raises(ValueError, cnf, [[]])
\end{minted}

\subsection{Assignments}
\label{sec:org3c79422}
A \emph{truth-assignment} is a map that assigns true/false values to variables
and literals. The functions that implement these assignments will be
explained in \S \ref{sec:org3162dd0}, but here we specify the type for
assignments.

We make a distinction between \emph{full assignments}, which assign all
(countably-many) variables to true/false, and \emph{partial assignments}, which
assign only some finite subset of variables to true/false. The type of full
assignments is best encoded as a function of type \(\mathbb{N}\rightarrow\)
\texttt{Bool}, while the partial assignments are best encoded as a dictionary of
\texttt{Variable} keys and \texttt{Bool} values.

For computations, we will be more concerned with partial assignments, so we
denote their type by \texttt{Assignment} and define it to be a type alias for
Mapping from Variables to Bools called \texttt{Assignment}. Note that
\href{https://docs.python.org/3/library/collections.abc.html\#collections.abc.Mapping}{collections.abc.Mappings} are the abstract base class for python's \texttt{dict}
container type.

\begin{minted}[frame=lines,label= (python3.9) (cnf.py) <<Assignment>>]{python}
Assignment = Mapping[Variable, Bool]  # defines a type alias
\end{minted}
\subsection{Helpful constants}
\label{sec:orgfb01940}
We introduce some constants for ease of writing functions in the sections
that follow.  However, it should be noted that these constants are intended
for internal use only, and are not to be called by a user of this
module. To enforce this restriction, we name these constants starting with
an underscore to prevent them from being imported in other
modules. Further, we wrap their types in the \texttt{typing.Final} type construct
to prevent these values from being re-assigned or overridden.

\begin{minted}[frame=lines,label= (python3.9) (cnf.py) <<constants>>]{python}
_TRUE_CLAUSE  : Final[Clause] = clause([TRUE])
_FALSE_CLAUSE : Final[Clause] = clause([FALSE])
_TRUE_CNF     : Final[Cnf]    = cnf([_TRUE_CLAUSE])
_FALSE_CNF    : Final[Cnf]    = cnf([_FALSE_CLAUSE])
\end{minted}

\section{Basic functions}
\label{sec:org95949d5}
We define some basic functions for manipulating terms of one type into another --
\begin{itemize}
\item In \S \ref{sec:orga0c0671}, we will define a function that negates
Lits. This is a function that ties a Lit \(l\) to its negation \(\neg
  l\).
\item In \S \ref{sec:org013fa51}, we will define a function that computes
the absolute value of a Lit by returning its underlying Variable (but
cast as a Lit). This is the function that ties the Lits \(\{l, \neg l\}\)
to their underlying Variable, denoted \(|l|\).
\item In \S \ref{sec:orga212744}, we will define a function that returns a set of
all Lits in a Cnf. This function will allows us to iterate over, or
assign to, every Lit in the Cnf.
\end{itemize}

\subsection{Negation of literals}
\label{sec:orga0c0671}
The negation function sends a literal \(l\) to \(\neg l\). For Bools, the
negation sends \texttt{TRUE} to \texttt{FALSE}, and \texttt{FALSE} to \texttt{TRUE}.

One important property of the negation function is that it is an involution
since \(\neg(\neg l) = l\), for any literal \(l\). It should also be noted
that our negation function has a severely restricted scope compared to the
mathematical operation of \((\neg)\), since our negation function can only
act on Lits and not on Clauses or Cnfs. The reason for this is that the
negation of a Clause is not in general a Clause, and the negation of a Cnf
is not in general a Cnf.

\begin{minted}[frame=lines,label= (python3.9) (cnf.py) <<neg>>]{python}
def neg(literal: Lit) -> Lit:
    <<neg-docstring>>
    if literal == TRUE:
	return FALSE
    if literal == FALSE:
	return TRUE
    return lit(-literal)
\end{minted}

Finally, we add a docstring and some unit tests for the negation function.

\begin{minted}[frame=lines,label= (python3.9) (cnf.py) <<neg-docstring>>]{python}
"""Negate a Lit.

This function is an involution.

Args:
   literal (:obj:`Lit`): a Lit formed from a nonzero integer or from a Bool.

Return:
   Return the Lit cast from the negative of ``literal``. If ``literal`` is of type Bool, then
      return ``TRUE`` for ``FALSE``, ``FALSE`` for ``TRUE``.

"""
\end{minted}


\begin{minted}[frame=lines,label= (python3.9) (test/test\_cnf.py) <<test\_neg>>]{python}
def test_neg():
    assert neg(lit(1))  == lit(-1)
    assert neg(lit(-1)) == lit(1)
    assert neg(lit(23)) == lit(-23)
    assert neg(TRUE)    == FALSE
    assert neg(FALSE)   == TRUE

    # Test for involution.
    assert neg(neg(lit(1)))  == lit(1)
    assert neg(neg(lit(-1))) == lit(-1)

    pytest.raises(ValueError, neg, 0)
\end{minted}

\subsection{Absolute value of literals}
\label{sec:org013fa51}
The \emph{absolute value} of a literal \(l\) is denoted \(|l|\) and is defined
as the literal formed by the underlying variable. For Bools, the absolute
value of a Bool is itself.

\begin{minted}[frame=lines,label= (python3.9) (cnf.py) <<absolute\_value>>]{python}
def absolute_value(literal: Lit) -> Lit:
    <<absolute_value-docstring>>
    if isinstance(literal, Bool):
	return literal
    return lit(abs(literal))
\end{minted}

We then add a docstring and some unit tests for the function.

\begin{minted}[frame=lines,label= (python3.9) (cnf.py) <<absolute\_value-docstring>>]{python}
"""Unnegated form of a Lit.

This function is idempotent.

Args:
   literal (:obj:`Lit`): a Lit formed from a nonzero integer.

Return:
   Check that ``literal`` is not of type Bool and then return the absolute value of ``literal``.
      If it is of type Bool, then return ``literal`` as is.
"""
\end{minted}

\begin{minted}[frame=lines,label= (python3.9) (test/test\_cnf.py) <<test\_absolute\_value>>]{python}
def test_absolute_value():
    assert absolute_value(lit(1)) == lit(1)
    assert absolute_value(lit(-1)) == lit(1)

    # Test for idempotence.
    assert absolute_value(absolute_value(lit(1))) == absolute_value(lit(1))
    assert absolute_value(absolute_value(lit(-1))) == absolute_value(lit(-1))

    # Test for Bools
    assert absolute_value(TRUE) == TRUE
    assert absolute_value(FALSE) == FALSE

    pytest.raises(ValueError, absolute_value, 0)
\end{minted}

\subsection{Literals in a Cnf}
\label{sec:orga212744}
This function returns the set of Lits in a given Cnf. We wish to keep these
sets immutable, and hence we return frozensets instead of regular sets. It
is computed simply by combining all the clauses and putting the Lits thus
gathered into a frozenset.

\begin{minted}[frame=lines,label= (python3.9) (cnf.py) <<lits>>]{python}
def lits(cnf_instance: Cnf) -> frozenset[Lit]:
    <<lits-docstring>>
    return frozenset.union(*cnf_instance)
\end{minted}

Lastly, we add in a docstring and a unit test for this function.
\begin{minted}[frame=lines,label= (python3.9) (cnf.py) <<lits-docstring>>]{python}
"""Return frozenset of all Lits that appear in a Cnf.

Args:
   cnf_instance (:obj:`Cnf`)

Return:
   A frozenset of all lits that appear in a Cnf.
"""
\end{minted}

\begin{minted}[frame=lines,label= (python3.9) (test/test\_cnf.py) <<test\_lits>>]{python}
def test_lits():
    assert lits(cnf([[1, -2],[3, TRUE], [FALSE]])) == frozenset({1, -2, 3, TRUE, FALSE})
\end{minted}
\section{Functions for simplification}
\label{sec:org2d8bb5f}
We define some functions for reducing/simplifying clauses and Cnfs --
\begin{itemize}
\item In \S \ref{sec:org7a3de96} we define a function for applying a set
of standard tautologies for simplifying a Clause.
\item In \S \ref{sec:org373350b} we define a function for applying a set
of standard tautologies for simplifying a Cnf.
\end{itemize}

Both these functions come in use especially after making assignments in
Cnfs. Selectively assigning some Lits to true/false results in Clauses and
Cnfs that contain Bools and can be simplified. The simplification functions
defined in this section help us do just that.

\subsection{Tautologically reduce clauses}
\label{sec:org7a3de96}
We can reduce a Clause using the following tautologies -- \[\top \vee c =
\top, \quad \bot \vee c = c,\quad c \vee \neg c = \top,\] where \(c\) is a
Clause. These are tautologies that feature disjunctions -- which is what
bind all the Lits together in a Clause.

We check for these tautologies on a case-by-case basis. It should be noted
that the order in which the cases are checked is important for the correct
behavior of this function.

Furthermore, the input type of this function is abstracted to \texttt{Set[Lit]},
of which \texttt{set[Lit]} and \texttt{frozenset[Lit]} are subtypes. This allows us to
use both sets and frozensets as valid inputs to this function.

Checking of the first two tautologies is straightforward. We describe here
the implementation of how the third tautology \((c \vee \neg c = \top)\) is
checked. Given a Lit-set \(\{l_i \mid i \in I\}\), for some finite indexing
set \(I\), we check that \(\{\neg l_i \mid i\in I\} \cap \{l_i \mid i\in
I\} = \emptyset\). If yes, then we return a Clause formed from \(\{l_i\mid
i\in I\}\). If not, then we return the Clause \(\top\).

We also ensure that this reduction (and our implementation of it in the
form of this function) are idempotent.

\begin{minted}[frame=lines,label= (python3.9) (cnf.py) <<tautologically\_reduce\_clause>>]{python}
def tautologically_reduce_clause(lit_set: Set[Lit]) -> Clause:
    <<tautologically_reduce_clause-docstring>>
    if TRUE in lit_set:
	return _TRUE_CLAUSE
    if lit_set == {FALSE}:
	return _FALSE_CLAUSE
    if FALSE in lit_set:
	lit_set -= _FALSE_CLAUSE
    if not set(map(neg, lit_set)).isdisjoint(lit_set):
	return _TRUE_CLAUSE
    return clause(lit_set)
\end{minted}

Having implemented the function, we now include a docstring and some unit
tests for it.

\begin{minted}[frame=lines,label= (python3.9) (cnf.py) <<tautologically\_reduce\_clause-docstring>>]{python}
r"""Reduce a Clause using various tautologies.

The order in which these reductions are performed is important. This function is idempotent.

Tautologies affecting Clauses:
   (⊤ ∨ c = ⊤)  (⊥ = ⊥)  (⊥ ∨ c = c)  (c ∨ ¬c = ⊤),
   where `x` is a Clause, `⊤` represents ``TRUE``, `⊥` represents ``FALSE``, and `∨` is
      disjunction.

Args:
   lit_set (:obj:`Set[Lit]`): an abstract set (a set or a frozenset) of Lits.

Return:
   The Clause formed by performing all the above-mentioned tautological reductions.
"""
\end{minted}

\begin{minted}[frame=lines,label= (python3.9) (test/test\_cnf.py) <<test\_tautologically\_reduce\_clause>>]{python}
def test_tautologically_reduce_clause():
    trc = tautologically_reduce_clause  # local alias for long function name
    assert trc(clause([lit(1), TRUE]))    == clause([TRUE])
    assert trc(clause([FALSE]))           == clause([FALSE])
    assert trc(clause([lit(1), FALSE]))   == clause([lit(1)])
    assert trc(clause([lit(1), lit(-1)])) == clause([TRUE])

    # Test for idempotence
    _clause = clause([lit(1), lit(-2), lit(3), lit(3)])
    assert trc(trc(_clause)) == trc(_clause)

    pytest.raises(ValueError, trc, set())
\end{minted}

\subsection{Tautologically reduce Cnfs}
\label{sec:org373350b}
We can reduce a Cnf using the following tautologies -- \[x \wedge \bot =
\bot,\quad \top \wedge x = x,\] where \(x\) is a Cnf. These are tautologies
that feature conjunctions -- which is what bind all the Clauses together in
a Cnf.

We check for these tautologies on a case-by-cases basis. It should be noted
that the order in which the cases are checked is important for the correct
behavior of this function.

The input type is abstracted to \texttt{Set[Set[Lit]]}. This means the input can
be one of the following types --
\begin{enumerate}
\item a frozenset of frozenset of Lits -- this is the preferred type and all
other inputs will be converted to this type by the function.
\item a frozenset of set of Lits -- this form is not possible because a
frozenset will only accept entries of hashable type.
\item a set of frozenset of Lits
\item a set of set of Lits -- this form is not possible because a set will
only accept entries hashable type.
\end{enumerate}

Given an input of the right type, the function first reduced all its
clauses using the\\
\texttt{tautologically\_reduce\_clause} function. In the case when the clause set
contains the TRUE Clause, we remove the TRUE clause and call our function
recursively on the reduced clause-set. This recursive call is guaranteed to
terminate because we always apply it to a clause set of smaller length each
time.

Lastly, we note that this function is idempotent.

\begin{minted}[frame=lines,label= (python3.9) (cnf.py) <<tautologically\_reduce\_cnf>>]{python}
def tautologically_reduce_cnf(clause_set: Set[Set[Lit]]) -> Cnf:
    <<tautologically_reduce_cnf-docstring>>
    clause_set_reduced: set[Clause]
    clause_set_reduced = set(map(tautologically_reduce_clause, clause_set))

    if _FALSE_CLAUSE in clause_set_reduced:
	return _FALSE_CNF
    if clause_set_reduced == _TRUE_CNF:
	return _TRUE_CNF
    if _TRUE_CLAUSE in clause_set_reduced:
	return tautologically_reduce_cnf(clause_set_reduced - _TRUE_CNF)
    return cnf(clause_set_reduced)
\end{minted}

Below, we specify the docstring and some unit tests for this function.

\begin{minted}[frame=lines,label= (python3.9) (cnf.py) <<tautologically\_reduce\_cnf-docstring>>]{python}
r"""Reduce a Cnf using various tautologies.

The order in which these reductions are performed is important. This function is idempotent. This
is a recursive function that is guaranteed to terminate.

Tautologies affecting Cnfs:
   (x ∧ ⊥ = ⊥)  (⊤ = ⊤)  (⊤ ∧ x = x),
   where `x` is a Cnf, `⊤` represents ``TRUE``, `⊥` represents ``FALSE``, and `∧` is conjunction.

Args:
   clause_set (:obj:`Set[Set[Lit]]`): an abstract set (set or frozenset) of abstract sets of Lits.

Return:
   The Cnf formed by first reducing all the clauses tautologically and then performing all the
   above-mentioned tautological reductions on the Cnf itself.
"""
\end{minted}

\begin{minted}[frame=lines,label= (python3.9) (test/test\_cnf.py) <<test\_tautologically\_reduce\_cnf>>]{python}
def test_tautologically_reduce_cnf():
    trc = tautologically_reduce_cnf  # local alias for long function name
    # cases where Clause reductions appear within Cnf reductions
    assert trc(cnf([[1, TRUE], [1, 2]]))  == cnf([[1, 2]])
    assert trc(cnf([[FALSE], [1, 2]]))    == cnf([[FALSE]])
    assert trc(cnf([[1, FALSE], [1, 2]])) == cnf([[1], [1, 2]])
    assert trc(cnf([[1, -1], [1, 2]]))    == cnf([[1, 2]])

    # cases where we might have two simultaneous clause reductions
    assert trc(cnf([[1, TRUE], [FALSE]]))    == cnf([[FALSE]])
    assert trc(cnf([[1, TRUE], [1, FALSE]])) == cnf([[1]])
    assert trc(cnf([[1, TRUE], [1, -1]]))    == cnf([[TRUE]])
    assert trc(cnf([[FALSE], [1, FALSE]]))   == cnf([[FALSE]])
    assert trc(cnf([[FALSE], [1, -1]]))      == cnf([[FALSE]])
    assert trc(cnf([[1, FALSE], [1, -1]]))   == cnf([[1]])

    # cases where we might have a cnf-related tautology
    assert trc(cnf([[1], [FALSE]])) == cnf([[FALSE]])
    assert trc(cnf([[TRUE]]))       == cnf([[TRUE]])
    assert trc(cnf([[1], [TRUE]]))  == cnf([[1]])

    # Test for idempotence.
    _cnf = cnf([[lit(1), lit(2)], [lit(-2)]])
    assert trc(trc(_cnf)) == trc(_cnf)

    pytest.raises(ValueError, trc, set())
    pytest.raises(ValueError, trc, frozenset())
\end{minted}

\section{Functions for assignment}
\label{sec:org3162dd0}
We define functions for applying truth-assignments to Lits, Clauses, and Cnfs.
\begin{itemize}
\item In \S \ref{sec:orgdde8b5a} we define a function that assigns a
given truth value to a variable in a Lit.
\item In \S \ref{sec:org920ac88} we define a function that assigns a
given truth value to a variable in a Clause.
\item In \S \ref{sec:org4137674} we define a function that assigns a given
truth value to a variable in a Cnf.
\item In \S \ref{sec:orgda79807} we define a function that applies an Assignment to Cnf.
\end{itemize}

We refer back to the \texttt{<<example>>} code-block in \S \ref{sec:orga8ba48d} for a
demonstration of the usage of the \texttt{assign} function defined in \S \ref{sec:orgda79807}.

Throughout this section, we will denote the assignment of \(a\) to an
object \(x\) by the notation \(x[a]\).

\subsection{Assign to variable in literals}
\label{sec:orgdde8b5a}
For a literal \(l\) and a variable \(v\), this function encodes the result
that the value \(l[v]\), i.e. the value obtained by setting \(v\) to \(\top\) is,
\begin{itemize}
\item \(\top\), if \(l = v\),
\item \(\bot\), if \(l = \neg v\),
\item \(l\), otherwise.
\end{itemize}

This function is idempotent because \(l[v][v] = l[v]\), for every literal
\(l\) and every variable \(v\).

\begin{minted}[frame=lines,label= (python3.9) (cnf.py) <<assign\_variable\_in\_lit>>]{python}
def assign_variable_in_lit(literal: Lit, variable_instance: Variable, boolean: Bool) -> Lit:
    <<assign_variable_in_lit-docstring>>
    if literal == variable_instance:
	return boolean
    if neg(literal) == variable_instance:
	return neg(boolean)
    return literal
\end{minted}

We complete the definition by adding in a docstring and some unit-tests for
the function.
\begin{minted}[frame=lines,label= (python3.9) (cnf.py) <<assign\_variable\_in\_lit-docstring>>]{python}
"""Assign Bool value to a Variable if present in Lit.

Replace all instances of ``variable_instance`` and its negation with ``boolean`` and its negation
respectively. Leave all else unchanged. This function is idempotent.

Args:
   literal (:obj:`Lit`)
   variable_instance (:obj:`Variable`)
   boolean (:obj:`Bool`): either ``TRUE`` or ``FALSE``.

Return:
   Lit formed by assigning ``variable_instance`` to ``boolean`` in ``literal``.
"""
\end{minted}


\begin{minted}[frame=lines,label= (python3.9) (test/test\_cnf.py) <<test\_assign\_variable\_in\_lit>>]{python}
def test_assign_variable_in_lit():
    avil = assign_variable_in_lit  # local alias for long function name

    assert avil(1, 1, TRUE)     == TRUE
    assert avil(1, 1, FALSE)    == FALSE
    assert avil(-1, 1, TRUE)    == FALSE
    assert avil(-1, 1, FALSE)   == TRUE
    assert avil(1, 2, TRUE)     == 1
    assert avil(TRUE, 1, TRUE)  == TRUE
    assert avil(FALSE, 1, TRUE) == FALSE

    # Test for idempotence
    assert avil(avil(1, 1, TRUE), 1, TRUE)  == avil(1, 1, TRUE)
    assert avil(avil(-1, 1, TRUE), 1, TRUE) == avil(-1, 1, TRUE)
\end{minted}

\subsection{Assign to variable in clauses}
\label{sec:org920ac88}
Next, we define a function that assigns variables to Bools wherever the
variable occurs in a clause. The function carries out the following computation --
\begin{align*}
  c[v] &= (l_1 \vee l_2 \vee \cdots \vee l_n)[v],\\
       &= l_1[v] \vee l_2[v] \vee \cdots \vee l_n[v],
\end{align*}
where \(c\) is a clause, \(v\) is a variable, and each \(l_i\) is a literal.

In the implementation of the function, we use \texttt{functools.partial} to define
\texttt{assign\_variable} to be the partial function \texttt{assign\_variable\_in\_lit(\_\_,
variable\_instance, boolean)}. This partial function takes a single Lit as
an input while freezing the other arguments in place.

\begin{minted}[frame=lines,label= (python3.9) (cnf.py) <<assign\_variable\_in\_clause>>]{python}
def assign_variable_in_clause(lit_set: Set[Lit], variable_instance: Variable, boolean: Bool) \
	-> Clause:
    <<assign_variable_in_clause-docstring>>
    assign_variable: Callable[[Lit], Lit]
    assign_variable = ft.partial(assign_variable_in_lit, variable_instance=variable_instance,
				 boolean=boolean)
    mapped_lits: set[Lit]
    mapped_lits = set(map(assign_variable, lit_set))

    return tautologically_reduce_clause(mapped_lits)
\end{minted}

We now add in a docstring and some unit-tests for the function we just defined.
\begin{minted}[frame=lines,label= (python3.9) (cnf.py) <<assign\_variable\_in\_clause-docstring>>]{python}
"""Assign Bool value to a Variable if present in Clause.

Replace all instances of ``variable_instance`` and its negation in ``lit_set`` with ``boolean``
and its negation respectively. Leave all else unchanged. Perform tautological reductions on the
Clause before returning results. This function is idempotent.

Args:
   lit_set (:obj:`Set[Lit]`): an abstract set (set or frozenset) of Lits.
   variable_instance (:obj:`Variable`)
   boolean (:obj:`Bool`): either ``TRUE`` or ``FALSE``.

Return:
   Tautologically-reduced Clause formed by assigning ``variable_instance`` to ``boolean`` in
      ``lit_set``.
"""
\end{minted}


\begin{minted}[frame=lines,label= (python3.9) (test/test\_cnf.py) <<test\_assign\_variable\_in\_clause>>]{python}
def test_assign_variable_in_clause():
    avic = assign_variable_in_clause  # local alias for long function name

    assert avic(clause([1, -2]), 1, TRUE)      == {TRUE}
    assert avic(clause([1, -2]), 1, FALSE)     == {-2}
    assert avic(clause([1, -2, -1]), 1, TRUE)  == {TRUE}
    assert avic(clause([1, -2, -1]), 1, FALSE) == {TRUE}
    assert avic(clause([1, -2]), 2, TRUE)      == {1}
    assert avic(clause([1, -2]), 2, FALSE)     == {TRUE}
    assert avic(clause([1, -2, -1]), 2, TRUE)  == {TRUE}
    assert avic(clause([1, -2, -1]), 2, FALSE) == {TRUE}

    # Test for idempotence
    _clause: Clause = clause([lit(1), lit(-2), lit(-1)])
    _var: Variable = variable(2)
    assert avic(avic(_clause, _var, FALSE), _var, FALSE) == avic(_clause, _var, FALSE)

    pytest.raises(ValueError, assign_variable_in_clause, [], 1, TRUE)
\end{minted}

\subsection{Assign to variable in Cnf}
\label{sec:org4137674}

\begin{minted}[frame=lines,label= (python3.9) (cnf.py) <<assign\_variable\_in\_cnf>>]{python}
def assign_variable_in_cnf(clause_set: Set[Set[Lit]], variable_instance: Variable, boolean: Bool) \
	-> Cnf:
    <<assign_variable_in_cnf-docstring>>
    assign_variable: Callable[[Clause], Clause]
    assign_variable = ft.partial(assign_variable_in_clause,
				 variable_instance=variable_instance,
				 boolean=boolean)

    mapped_clauses: set[Clause]
    mapped_clauses = set(map(assign_variable, clause_set))

    return tautologically_reduce_cnf(mapped_clauses)
\end{minted}

\begin{minted}[frame=lines,label= (python3.9) (cnf.py) <<assign\_variable\_in\_cnf-docstring>>]{python}
"""Assign Bool value to a Variable if present in Cnf.

Replace all instances of ``variable_instance`` and its negation in ``clause_set`` with
``boolean`` and its negation respectively. Leave all else unchanged. Perform tautological
reductions on the Cnf before returning results. This function is idempotent.

Args:
   clause_set (:obj:`Set[Set[Lit]]`): an abstract set (set or frozenset) of abstract sets of
      Lits.
   variable_instance (:obj:`Variable`)
   boolean (:obj:`Bool`): either ``TRUE`` or ``FALSE``.

Return:
   Tautologically-reduced Cnf formed by assigning ``variable_instance`` to ``boolean`` in
      ``clause_set``.
"""
\end{minted}


\begin{minted}[frame=lines,label= (python3.9) (test/test\_cnf.py) <<test\_assign\_variable\_in\_cnf>>]{python}
def test_assign_variable_in_cnf():
    avic = assign_variable_in_cnf  # local alias for long function name

    assert avic(cnf([[1, -2], [-1, 3]]), 1, TRUE)  == cnf([[3]])
    assert avic(cnf([[1, -2], [-1, 3]]), 1, FALSE) == cnf([[-2]])

    # Test for idempotence.
    _cnf = cnf([[1, -2], [-1, 3]])
    assert avic(avic(_cnf, 1, FALSE), 1, FALSE) == avic(_cnf, 1, FALSE)

    pytest.raises(ValueError, avic, [[]], 1, TRUE)
\end{minted}

\subsection{Assign}
\label{sec:orgda79807}

\begin{minted}[frame=lines,label= (python3.9) (cnf.py) <<assign>>]{python}
def assign(cnf_instance: Cnf, assignment: Assignment) -> Cnf:
    <<assign-docstring>>
    cnf_copy: frozenset[Clause] = cnf_instance.copy()
    for variable_instance, boolean in assignment.items():
	cnf_copy = assign_variable_in_cnf(cnf_copy, variable_instance, boolean)
    return tautologically_reduce_cnf(cnf_copy)
\end{minted}

\begin{minted}[frame=lines,label= (python3.9) (cnf.py) <<assign-docstring>>]{python}
"""Assign Bool values to Variables if present in Cnf.

For each Variable (key) in ``assignment``, replace the Variable and its negation in
``cnf_instance`` using the Bool (value) in ``assignment``. The final output is always
tautologically reduced. This function is idempotent.

Args:
   cnf_instance (:obj:`Cnf`)
   assignment (:obj:`Assignment`): a dict with keys being Variables to be replaced and values
      being Bools that the Variables are to be assigned to.  The ``assignment`` dict need not be
      complete and can be partial.

Edge case:
   An empty assignment dict results in ``cnf_instance`` simply getting topologically reduced.

Return:
   Tautologically-reduced Cnf formed by replacing every key in the ``assignment`` dict (and those
      keys' negations) by corresponding Bool values.
"""
\end{minted}


\begin{minted}[frame=lines,label= (python3.9) (test/test\_cnf.py) <<test\_assign>>]{python}
def test_assign():
    assert assign(cnf([[1, -2], [-1, 3]]), {1: TRUE})                     == cnf([[3]])
    assert assign(cnf([[1, -2], [-1, 3]]), {1: TRUE, 2: FALSE})           == cnf([[3]])
    assert assign(cnf([[1, -2], [-1, 3]]), {1: TRUE, 2: FALSE, 3: FALSE}) == cnf([[FALSE]])
    assert assign(cnf([[TRUE]]), {1: TRUE})                               == cnf([[TRUE]])
    assert assign(cnf([[TRUE]]), {})                                      == cnf([[TRUE]])
    assert assign(cnf([[FALSE]]), {})                                     == cnf([[FALSE]])
    assert assign(cnf([[1]]), {})                                         == cnf([[1]])

    # Test for idempotence.
    _cnf = cnf([[1, -2], [-1, 3]])
    assert assign(assign(_cnf, {1: TRUE}), {1: TRUE}) == assign(_cnf, {1: TRUE})

    pytest.raises(ValueError, assign, [[]], {1: TRUE})
\end{minted}

\subsection{Equi-satisfiability of Cnfs}
\label{sec:org24e2fdc}
\todo{add a new function to sat.py}

\section{Standalone script run commands}
\label{sec:org682198f}
\begin{minted}[frame=lines,label= (python3.9) (cnf.py) <<standalone\_run\_commands>>]{python}
logger.info('Cnfs can be constructed using the cnf() function.')
logger.info('>>> cnf([[1, -2], [3, 500]])')
logger.info(cnf([[1, -2], [3, 500]]))
\end{minted}

\section{Tangling}
\label{sec:orgd7f624b}
We now tangle all the code blocks into a single file (\texttt{cnf.py})
following the overview laid out in \S \ref{sec:orgd32c73b}.
We similarly tangle all the unit-tests into a single file
(\texttt{test/test\_cnf.py}) using the following layout.
\begin{minted}[frame=lines,label= (python3.9) (test/test\_cnf.py) <<test\_cnf.py-full>>]{python}
#!/usr/bin/env python3.9
import pytest
from cnf import *

<<test_variable>>
<<test_Bool>>
<<test_lit>>
<<test_clause>>
<<test_cnf>>

<<test_neg>>
<<test_absolute_value>>
<<test_lits>>

<<test_tautologically_reduce_clause>>
<<test_tautologically_reduce_cnf>>

<<test_assign_variable_in_lit>>
<<test_assign_variable_in_clause>>
<<test_assign_variable_in_cnf>>
<<test_assign>>
\end{minted}


Finally, we can run all the tests on our functions as follows:
\begin{minted}[frame=lines,label= (sh) (scratch)]{sh}
mypy cnf.py | cat
py.test
\end{minted}
\end{document}